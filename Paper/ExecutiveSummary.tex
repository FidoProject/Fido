\documentclass[letterpaper,12pt]{article}
\usepackage[english]{babel}
\usepackage[utf8]{inputenc}
\usepackage[nottoc]{tocbibind}
\usepackage[margin=1.5in]{geometry}
%\usepackage{mathptmx}
\usepackage[activate={true,nocompatibility},final,
			tracking=true,kerning=true,spacing=true,
			factor=1100,stretch=10,shrink=10]{microtype}
\usepackage{amsmath,setspace} \long\def\/*#1*/{}
\doublespacing

\begin{document}

\/*

The Executive Summary on its own, separate from the Research Report, should convey the essence of your project and should be understood by someone without scientific expertise. Do not simply replicate what you wrote in your Abstract. The summary should clearly present three content areas - the question asked, the methods used and the lessons learned and must be written in layperson (non-specialist) language. NOTE: The summary will be used to explain your project to the general public and in preparing press releases for the media.

*/

\begin{center}
	{\Large
	\textbf{Fido: a General Robotic Control System using Reinforcement Learning with Limited Feedback}}\\
	\vspace{1cm}
	{\large \textbf{Executive Summary}}
\end{center}

\noindent

The purpose of this project was to develop a control system for general robot operation through reinforcement learning.   Any living organism with neural capability can be reduced to a black box system, with inputs such as senses and outputs such as motor control.

\end{document}